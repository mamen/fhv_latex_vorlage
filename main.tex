\documentclass[a4paper,12pt,twoside]{scrreprt}
% Autor der Vorlage: Klaus Rheinberger, FH Vorarlberg
% 2017-02-20

%% Hilfe: z.B.
% empfohlener Einstieg: http://latex.tugraz.at/  
% https://de.wikibooks.org/wiki/LaTeX-Kompendium:_Schnellkurs:_Erste_Schritte
% https://de.wikibooks.org/wiki/LaTeX-Kompendium:_Schnellkurs
% https://de.wikibooks.org/wiki/LaTeX-Kompendium

%% Pakete:
% Der Befehl \usepackage[latin9]{inputenc} ermöglicht die direkte Angabe von Umlauten. Übrigens lässt sich so auch das Euro-Zeichen direkt eingeben. Auf Betriebssystemen, wie zum Beispiel allen neueren Linux-Distributionen, verwendet man statt \usepackage[latin9]{inputenc} besser \usepackage[utf8]{inputenc}, auf Applesystemen verwendet man \usepackage[macce]{inputenc} (oder das für ältere Modelle gültige \usepackage[applemac]{inputenc}).
\usepackage[utf8]{inputenc}
\usepackage[T1]{fontenc}    % Silbentrennung bei Sonderzeichen
\usepackage{graphicx}       % Bilder einbinden
\usepackage[ngerman]{babel} % Deutsche Sprachanpassungen
\usepackage{csquotes}       % When using babel or polyglossia with biblatex, loading csquotes is recommended to ensure that quoted texts are typeset according to the rules of your main language.
\usepackage{acronym}  % für optionales Abkürzungsverzeichnis
\usepackage{eurosym}  % z. B. \EUR{12345,68}
\usepackage[linktocpage=true]{hyperref} % Links z. B. \href{https://www.wikibooks.org}{Wikibooks home}
\usepackage[bindingoffset=8mm]{geometry}% Bindeverlust von 8mm einbeziehen. Mit dem geometry-Paket können Sie die Ränder auch ganz individuell anpassen.
\usepackage{caption} % Abbildungslegenden
\usepackage{import}
\captionsetup{format=hang, justification=raggedright}

\usepackage[style=alphabetic,citestyle=alphabetic,backend=bibtex]{biblatex}   % Literaturverweise
%\usepackage[style=numeric,citestyle=numeric,backend=biber]{biblatex}
% biblatex comes with a variety of built-in bibliography/citation style families (numeric, alphabetic, authoryear, authortitle, verbose), and there's a growing number of custom styles:
% https://de.sharelatex.com/learn/Biblatex_citation_styles
% https://de.sharelatex.com/learn/Biblatex_bibliography_styles

% Bibtex-Datei
\addbibresource{references.bib}


%% Einstellungen
\setcounter{secnumdepth}{4}
\setcounter{tocdepth}{4}   % Tiefe der Gliederung im In haltsverzeichnis


\begin{document}

% Sperrvermerkseite (OPTIONAL)
\import{pages/general/}{sperrvermerk.tex}

% Titelblatt
\cleardoublepage
\import{pages/general/}{titlepage.tex}

% Widmung (OPTIONAL)
\newpage
\import{pages/general/}{widmung.tex}

% Kurzreferat
\newpage
\import{pages/general/}{kurzrefarat.tex}

% Abstract
\newpage
\import{pages/general/}{abstract.tex}

% Vorwort (OPTIONAL)
\newpage
\import{pages/general/}{vorwort.tex}

% Inhaltsverzeichnis
\cleardoublepage
\tableofcontents

% Abbildungsverzeichnis
\clearpage
\phantomsection
\addcontentsline{toc}{chapter}{Abbildungsverzeichnis}
\listoffigures

% Tabellenverzeichnis
\clearpage
\phantomsection
\addcontentsline{toc}{chapter}{Tabellenverzeichnis}
\listoftables

% Abkürzungsverzeichnis (OPTIONAL)
\clearpage
\import{pages/general/}{abkuerzungsverzeichnis.tex}

% Kapitel
\import{pages/chapters/}{chapter1.tex}
\import{pages/chapters/}{chapter2.tex}
\import{pages/chapters/}{chapter3.tex}

% Literaturverzeichnis
\clearpage
\phantomsection
\addcontentsline{toc}{chapter}{Literaturverzeichnis}
\printbibliography

% Anhang
\import{pages/general/}{anhang.tex}

% Eidesstattliche Erklärung
\import{pages/general/}{eidesstattliche_erklaerung.tex}

\end{document}
